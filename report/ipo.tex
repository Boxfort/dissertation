\vspace{0.3cm}

\section*{Initial Project Overview}\\

\vspace{0.3cm}

\section*{SOC10101 Honours Project (40 Credits)} \\

\vspace{0.3cm}

\section*{Title of Project: A Comparison of Machine Learning Algorithms to Detect Network Intrusions} \\

\vspace{0.3cm}

\section*{Overview of Project Content and Milestone}
\noindent For this project, the main objective is to research different machine learning strategies for detecting network intrusions and to develop a piece of software which can obtain metrics from these algorithms. The focus of this project will be on a comparison of two-stage classifiers versus single-stage classifiers for detecting and classifying network intrusions. The dataset upon which these methods will be tested is the KDD Cup 1999 dataset, with the possibility of testing upon other data sets should time allow it. \\


\noindent The different strategies for machine learning will be gathered by reading relevant research papers and articles within the field of machine learning and network intrusion detection and selecting appropriate algorithms/strategies which are applicable for the chosen data set, and which are also feasible to implement within the timescale. \\


\noindent The software will be a desktop application and will take either a predetermined algorithm or a user submitted algorithm, and a data set of network traffic. The software should then run the algorithm and extract metrics from it such as rates for false positives, true positives, false negatives, true negatives, overall accuracy of classification, etc. These results can then be used to plot graphs and charts to visualise this information to a user in a useful way. \\


\noindent A dissertation will be delivered at the end of the project and should contain a detailed design and plan of the software as well as complete testing strategy and results. Also included in the final report should be a comparison of several of the implemented algorithms to determine which is the most suitable for use if any at all. \\

\noindent A list of milestones for this project goes as the following:
\begin{multicols}{2}
\begin{itemize}
\item{Initial Project Overview Submitted}
\item{Relevant Algorithms Selected}
\item{Project timescale Completed}
\item{Literature Review Complete}
\item{Software Design Completed}
\item{Algorithms Implemented}
\item{Software Implemented}
\item{Software Testing Plan}
\item{Software Tested}
\item{Algorithm Comparison Completed}
\item{Dissertation Written}
\item{Poster Presentation Completed}
\item{Project Submitted}
\end{itemize}
\end{multicols}

\section*{The Main Deliverable(s):}
\noindent A list of the main deliverables for the project is as follows:
\begin{multicols}{2}
	\begin{itemize}
	\item{Initial Project Overview}
	\item{Gantt Chart}
	\item{Literary Review}
	\item{Interim Report}
	\item{Requirement Specification}
	\item{Software Design Document}
	\item{Software Test Plan}
	\item{Software Test Results}
	\item{Software Implementation}
	\item{Algorithm Implementations}
	\item{Meeting Diary}
	\item{Algorithm Experimental Results}
	\item{Algorithm Comparisons}
	\item{Software User Documentation}
	\item{Dissertation}
	\item{Poster Presentation
	\end{itemize}
\end{multicols} \\


\section*{The Target Audience for the Deliverable(s):}
\noindent The target audience for this project could include, machine learning and network intrusion researchers, and computer science students. Researchers may find the comparison of algorithms to be highly useful when carrying out preliminary research and could save time on selecting or discounting an algorithm. Computer science students may also find this project of use for experimenting with different network intrusion methods and different machine learning methods, giving them an insight on what they can be used for and how effective they are. \\


\section*{The Work to be Undertaken:}
\noindent During this project, the work which must be undertaken is first extensive research of the subject area and collection of relevant sources. The project must then be planned and a timeline of work set out to be completed, with deadlines for each deliverable. Algorithms such as k-nearest neighbour, Artificial neural network, negative selection genetic algorithm, etc, will be implemented, compared and contrasted, specifically the performance of single stage against two-stage classifiers using these algorithms. At the same time as implementing these algorithms a requirement specification and then a design document will be created for the software package as well as a test plan. The design will then be implemented and then be tested according to the test plan. Once fully tested and proven correct the software can then be used to compare the algorithms and obtain metrics from then. The final dissertation will then be written which will include an analysis of the results for each algorithm. \\


\section*{Additional Information / Knowledge Required:}
\noindent Additional research is required on network intrusion and two stage classifiers to best select the methods which will be implemented and explored. Research into similar software products such as WEKA ("Weka 3 - Data Mining with Open Source Machine Learning Software in Java", 2017) will also be conducted to source ideas and to see in which areas these pieces of software are lacking. Research on each individual algorithm to be implemented will also be required to ensure a correct implementation. And finally, an investigation into relevant libraries which may be used to assist with GUI creation, Graphing, and algorithm implementations. \\


\section*{Information Sources that Provide a Context for the Project:}
\noindent 1. Tsai, C. F., Hsu, Y. F., Lin, C. Y., & Lin, W. Y. (2009). Intrusion detection by machine learning: A review. Expert Systems with Applications, 36(10), 11994-12000. \\

\noindent 2. Sommer, R., & Paxson, V. (2010, May). Outside the closed world: On using machine learning for network intrusion detection. In Security and Privacy (SP), 2010 IEEE Symposium on (pp. 305-316). IEEE. \\

\noindent 3. Denning, D. E. (1987). An intrusion-detection model. IEEE Transactions on software engineering, (2), 222-232. \\

\noindent 4. Powers, S. T., & He, J. (2008). A hybrid artificial immune system and Self Organising Map for network intrusion detection. Information Sciences, 178(15), 3024-3042. \\

\noindent 5. Shon, T., & Moon, J. (2007). A hybrid machine learning approach to network anomaly detection. Information Sciences, 177(18), 3799-3821. \\

\noindent 6. Mukkamala, S., Janoski, G., & Sung, A. (2002). Intrusion detection using neural networks and support vector machines. In Neural Networks, 2002. IJCNN'02. Proceedings of the 2002 International Joint Conference on (Vol. 2, pp. 1702-1707). IEEE. \\

\noindent 7. Frank, J. (1994, October). Artificial intelligence and intrusion detection: Current and future directions. In Proceedings of the 17th national computer security conference (Vol. 10, pp. 1-12). \\

\noindent 8. Weka 3 - Data Mining with Open Source Machine Learning Software in Java. (2017). Cs.waikato.ac.nz. Retrieved 21 September 2017, from http://www.cs.waikato.ac.nz/ml/weka/ \\

\section*{The Importance of the Project:}
\noindent This project has importance as there has not been many papers directly comparing implementations of different machine learning approaches to network intrusion detection. While there are software packages which deal with gaining metrics from and comparing algorithms there is not one which is focused solely on network intrusion detection. Making a piece of software which is focused on one area of research may prove to provide greater insights, through more focussed results or through ease of use regarding comparing single and multiple stage classifiers, whereas a more complicated piece of software may take a long time to become acquainted with and to produce results. \\


\section*{The Key Challenge(s) to be Overcome:}
\noindent The key challenges to be overcome in this project are the implementations of the machine learning algorithms themselves. This is due to a personal lack of experience in implementing machine learning algorithms. Experience is also lacked in understanding formal descriptions of algorithms which may hinder my understanding of techniques when reading research papers. Another challenge will be creating a method of accommodating algorithms by creating interfaces which they will communicate with the main piece of software allowing for any algorithm to be entered by a user.
